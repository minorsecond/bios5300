\documentclass[11pt,]{article}
\usepackage[left=1in,top=1in,right=1in,bottom=1in]{geometry}
\newcommand*{\authorfont}{\fontfamily{phv}\selectfont}
\usepackage[]{mathpazo}


  \usepackage[T1]{fontenc}
  \usepackage[utf8]{inputenc}



\usepackage{abstract}
\renewcommand{\abstractname}{}    % clear the title
\renewcommand{\absnamepos}{empty} % originally center

\renewenvironment{abstract}
 {{%
    \setlength{\leftmargin}{0mm}
    \setlength{\rightmargin}{\leftmargin}%
  }%
  \relax}
 {\endlist}

\makeatletter
\def\@maketitle{%
  \newpage
%  \null
%  \vskip 2em%
%  \begin{center}%
  \let \footnote \thanks
    {\fontsize{18}{20}\selectfont\raggedright  \setlength{\parindent}{0pt} \@title \par}%
}
%\fi
\makeatother




\setcounter{secnumdepth}{0}

\usepackage{longtable,booktabs}

\usepackage{graphicx,grffile}
\makeatletter
\def\maxwidth{\ifdim\Gin@nat@width>\linewidth\linewidth\else\Gin@nat@width\fi}
\def\maxheight{\ifdim\Gin@nat@height>\textheight\textheight\else\Gin@nat@height\fi}
\makeatother
% Scale images if necessary, so that they will not overflow the page
% margins by default, and it is still possible to overwrite the defaults
% using explicit options in \includegraphics[width, height, ...]{}
\setkeys{Gin}{width=\maxwidth,height=\maxheight,keepaspectratio}

\title{The Effects of Heavy Drinking, Smoking, Anxiety and Depression on Stroke
Odds: A Cross-Sectional Study  }



\author{\Large Robert Wardrup\vspace{0.05in} \newline\normalsize\emph{}   \and \Large Elizabeth Murphy\vspace{0.05in} \newline\normalsize\emph{}   \and \Large Natalie Requinez\vspace{0.05in} \newline\normalsize\emph{}   \and \Large Joshua Averitt\vspace{0.05in} \newline\normalsize\emph{}  }


\date{}

\usepackage{titlesec}

\titleformat*{\section}{\normalsize\bfseries}
\titleformat*{\subsection}{\normalsize\itshape}
\titleformat*{\subsubsection}{\normalsize\itshape}
\titleformat*{\paragraph}{\normalsize\itshape}
\titleformat*{\subparagraph}{\normalsize\itshape}


\usepackage{natbib}
\bibliographystyle{apa}
\usepackage[strings]{underscore} % protect underscores in most circumstances



\newtheorem{hypothesis}{Hypothesis}
\usepackage{setspace}

\makeatletter
\@ifpackageloaded{hyperref}{}{%
\ifxetex
  \PassOptionsToPackage{hyphens}{url}\usepackage[setpagesize=false, % page size defined by xetex
              unicode=false, % unicode breaks when used with xetex
              xetex]{hyperref}
\else
  \PassOptionsToPackage{hyphens}{url}\usepackage[unicode=true]{hyperref}
\fi
}

\@ifpackageloaded{color}{
    \PassOptionsToPackage{usenames,dvipsnames}{color}
}{%
    \usepackage[usenames,dvipsnames]{color}
}
\makeatother
\hypersetup{breaklinks=true,
            bookmarks=true,
            pdfauthor={Robert Wardrup () and Elizabeth Murphy () and Natalie Requinez () and Joshua Averitt ()},
             pdfkeywords = {Stroke, Alcohol, Smoking, Anxiety, Depression},  
            pdftitle={The Effects of Heavy Drinking, Smoking, Anxiety and Depression on Stroke
Odds: A Cross-Sectional Study},
            colorlinks=true,
            citecolor=blue,
            urlcolor=blue,
            linkcolor=magenta,
            pdfborder={0 0 0}}
\urlstyle{same}  % don't use monospace font for urls

% set default figure placement to htbp
\makeatletter
\def\fps@figure{htbp}
\makeatother



% add tightlist ----------
\providecommand{\tightlist}{%
\setlength{\itemsep}{0pt}\setlength{\parskip}{0pt}}

\begin{document}
	
% \pagenumbering{arabic}% resets `page` counter to 1 
%
% \maketitle

{% \usefont{T1}{pnc}{m}{n}
\setlength{\parindent}{0pt}
\thispagestyle{plain}
{\fontsize{18}{20}\selectfont\raggedright 
\maketitle  % title \par  

}

{
   \vskip 13.5pt\relax \normalsize\fontsize{11}{12} 
\textbf{\authorfont Robert Wardrup} \hskip 15pt \emph{\small }   \par \textbf{\authorfont Elizabeth Murphy} \hskip 15pt \emph{\small }   \par \textbf{\authorfont Natalie Requinez} \hskip 15pt \emph{\small }   \par \textbf{\authorfont Joshua Averitt} \hskip 15pt \emph{\small }   

}

}








\begin{abstract}

    \hbox{\vrule height .2pt width 39.14pc}

    \vskip 8.5pt % \small 

\noindent Though numerous studies have investigated the relationships between
lifestyle risk factors and stroke outcome, few have done so while
controlling as age and race/ethnicity range. Utilizing quasi-binomial
logistic regression, we show that the odds of having had a stroke
increase with being a daily smoker, days of anxiety, and having a
depression diagnosis.


\vskip 8.5pt \noindent \emph{Keywords}: Stroke, Alcohol, Smoking, Anxiety, Depression \par

    \hbox{\vrule height .2pt width 39.14pc}



\end{abstract}


\vskip 6.5pt


\noindent  \section{Introduction}\label{introduction}

Strokes nearly affect 800,000 people in the U.S every year and is the
fifth leading cause of death in the U.S
\citep{AmericanStrokeAssociation2017}. Despite increased efforts on
education and general public knowledge about stroke and other
cardiovascular risk diseases, the prevalence rates have shown little
improvement within the past ten years \citep{Mozaffarian2015}. Previous
research has shown that a variety of controllable risk factors that are
strongly associated with stroke outcome. Although behaviors such as
smoking and poor diet have been well-known risk factors, other factors
such as stress, lack of sleep, depression, and anxiety still have
unclear associations {[}(Risk Factors Stroke Association,
\citep{AmericanStrokeAssociation2017a}.

Stress, lack of sleep, and mental disorders such as depression and
anxiety are issues prominent throughout the United States. Anxiety is
the most common mental illness and approximately 18\% of the U.S
population is affected while depression affects nearly 7\% of adults
\citep{AnxietyandDepressionAssociationofAmerica2017}. Those who are
diagnosed with anxiety also have a 50\% chance of being diagnosed with
depression and vice versa
\citep{AnxietyandDepressionAssociationofAmerica2017}. Furthermore,
approximately, 30\% of Americans suffer from lack of sleep
\citep{CentersforDiseaseControlandPrevention2015}. Although average
stress levels have decreased since the recession, the average stress
scores for American adults is still at moderate levels and is increased
with lower household incomes
\citep{AmericanPsychologicalAssociation2015}. Each of these factors has
a considerable significance to public health as issues on their own. If
someone is suffering from multiple risk factors, the risk of stroke
could be significant and possibly result in death.

The goal of this study is to test the if there is an association between
the different risk factors, particularly with stress, lack of sleep,
depression, and anxiety, with the outcome stroke. This will be compared
to previous trends in the same survey data to see if results are
consistent and hopefully contribute additional knowledge. This will be
determined using a chi squared analysis and logistics regressions
collected from the BRFSS National Survey. The BRFSS National Survey
collects data from 50 states, D.C, and the even throughout the
territories collecting telephone survey data. It is in partnership with
the CDC and has been useful for providing large quantities of health
related data \citep{CentersforDiseaseControlandPrevention2014}. The
results of this study will be compared to previous trends in the same
survey data to see if results are consistent and hopefully contribute
additional knowledge to help prevent stroke and improve quality of life.

\section{Methods}\label{methods}

\subsection{Josh's Text}\label{joshs-text}

\subsection{Survey Weighting}\label{survey-weighting}

The BRFSS survey data were weighted using the raking method, which is a
two-part methodology to help insure unbiased results by accounting for
noncoverage and nonresponse bias and forcing the total number of cases
to equal the population estimate of each state in the United States
\citep{CentersforDiseaseControlandPrevention2007a}. Raking works by
repeatedly adjusting weight across a set of selected variables until the
weights converge and the survey population totals are equal to the
census population totals for each selected variable \citep{Fricker1993}.

\subsection{Dependent and Independent
Variables}\label{dependent-and-independent-variables}

The dependent variable, stroke outcome, was recoded to a binary variable
for use in a logistic regression model. Depression diagnosis and heavy
drinker were also recoded into a binary variable. For each variable used
in the analysis, \emph{don't know / unsure} and \emph{refused} responses
were dropped. Table 1 lists the variables included in the model.

\begin{longtable}[]{@{}ll@{}}
\caption{Dependent and Independent Variables Used in Logistic Regression
Model}\tabularnewline
\toprule
Variable Name & Variable Type\tabularnewline
\midrule
\endfirsthead
\toprule
Variable Name & Variable Type\tabularnewline
\midrule
\endhead
Stroke Diagnosis & Dichotomous (DV)\tabularnewline
Heavy Alcohol Drinker & Dichotomous\tabularnewline
Depression Diagnosis & Dichotomous\tabularnewline
Sex & Dichotomous\tabularnewline
Days Anxious & Continuous\tabularnewline
Daily Sleep Hours & Continuous\tabularnewline
Smoker Status & Categorical\tabularnewline
Race/Ethnicity & Categorical\tabularnewline
Age Group & Categorical\tabularnewline
\bottomrule
\end{longtable}

A subject was designated as a heavy drinker if they were either an adult
male who reported consuming more than 14 drinks per week or an adult
female who reported consuming more than 7 drinks per week. This value
was calculated after asking the subjects, ``during the past 30 days, on
the days when you drank, about how many drinks did you drink on
average?'' and ``during the past 30 days, how many days per week or
month did you have at least one drink of any alcoholic beverage such as
beer, wine, a malt beverage or liquor?''
\citep{CentersforDiseaseControlandPrevention2007}.

\subsection{Logistic Regression Model}\label{logistic-regression-model}

The first model used in calculating the logistic regression utilized the
standard binomial function to predict the log odds of binary outcome
\(k\). The binomial formula, is given as
\[P(k)={n \choose k}p^{k}(1-p)^{n-k},\] where \(k\) represents a stroke
outcome, \(n\) represents the sample size, and \(p\) is the probability
that a stroke will occur. {[}\textbf{Explain overdispersion}{]} Before
being checked for overdispersion by calculating \(\phi\), given by the
formula
\[\phi = \frac{1}{(n-p-1)}\sum_{i=1}^{n} (y_{i} -\hat{y} _{i})^{2} 
/ \hat{y} _{i}.\] A threshold of \(\phi > 1\), was used to determine if
the data were overdispersed, possibly leading to unstable estimates. To
account for the overdispersion, \(\phi\) was included as a model
parameter, giving the formula
\[P(k)={n \choose k}p(p+k\phi)^{k-1}(1-p-k\phi)^{n-k}.\]
{[}\textbf{Explain the quasi-binomial family more}{]}

\section{Results}\label{results}

The weighted number of strokes in the United States in 2016 was
8,020,080, representing approximately 2.5\% of the total 2016 United
States population. Out of these, 3.20\% were female and 3.13\% were
male.

\begin{longtable}[]{@{}lll@{}}
\caption{Proportion of Stroke Outcomes in Males and
Females}\tabularnewline
\toprule
Diagnosis & Male & Female\tabularnewline
\midrule
\endfirsthead
\toprule
Diagnosis & Male & Female\tabularnewline
\midrule
\endhead
No Stroke & 96.87\% & 96.80\%\tabularnewline
Stroke & 3.13\% & 3.20\%\tabularnewline
\bottomrule
\end{longtable}

A chi-squared test of independence revealed that there there is not
statistically significant evidence that stroke outcome and sex is
independent from one another at an alpha level of 0.05, \(\chi^2\) (3,
\(N\)=486,303) = 1.73, \(p\)=0.48.

The proportion of subjects who have suffered from a stroke by smoker
status is shown in Table 3, and visually in Figure 2. The percentage of
subjects who have had a stroke is highest for the subjects who smoke
daily, followed by subjects who are former smokers, followed by subjects
who smoke some days. A possible explanation for the lower percentage of
stroke in subjects who smoke some days than subjects who are former
smokers is that stroke victims may simply have ceased smoking after
having a stroke.

\begin{longtable}[]{@{}lll@{}}
\caption{Proportion of Stroke Outcomes and Smoker Status}\tabularnewline
\toprule
Smoker Status & No Stroke & Stroke\tabularnewline
\midrule
\endfirsthead
\toprule
Smoker Status & No Stroke & Stroke\tabularnewline
\midrule
\endhead
Never Smoked & 97.8\% & 2.2\%\tabularnewline
Former Smoker & 95.4\% & 4.6\%\tabularnewline
Smokes Some Days & 96.2\% & 3.8\%\tabularnewline
Smokes Daily & 95.0\% & 5.0\%\tabularnewline
\bottomrule
\end{longtable}

The proportions shown in Table 3 are depicted as a mosaic plot in Figure
2. The mosaic plot is a graphical depiction the proportions within each
table cell, shaded by the difference from the expected observation. Blue
depicts a higher than expected number of observations for that cell, and
red depicts a lower than expected number of observations for that cell.
As shown in Figure 2, there were a greater than expected number of
non-smokers that have never had a stroke, and a greater than expected
number of smokers and former smokers who have had a stroke.

\begin{figure}
\centering
\includegraphics{report_files/figure-latex/figWithCaption3-1.pdf}
\caption{Proportion of Stroke Outcomes and Smoker Status}
\end{figure}

A chi-squared test of independence revealed that there is statistically
significant evidence that stroke outcome and smoking status were not
independent from one another, \(\chi^2\) (3, \(N\)=486,303) = 2352.5,
\(p <\) 0.001.

The mean hours of sleep for those who had a stroke was 6.99 hours and
6.98 for those who did not have a stroke.

\begin{longtable}[]{@{}lll@{}}
\caption{Mean Values of Selected Risk Factors}\tabularnewline
\toprule
Factor & Outcome & Mean\tabularnewline
\midrule
\endfirsthead
\toprule
Factor & Outcome & Mean\tabularnewline
\midrule
\endhead
Days Anxious & No Stroke & 4.79\tabularnewline
& Stroke & 8.05\tabularnewline
Nightly Sleep Hours & No Stroke & 6.98\tabularnewline
& Stroke & 6.99\tabularnewline
Number Drinks Weekly & No Stroke & 79.07\tabularnewline
& Stroke & 40.31\tabularnewline
\bottomrule
\end{longtable}

\subsection{Logistic Regression
Model}\label{logistic-regression-model-1}

The binomial logistic regression model resulted in a \(\phi\) value of
2.52, indicating that there is overdispersion in the estimates and the
quasi-binomial function should be used instead.

The results of the quasi-binomial logistic regression (Table 4) indicate
that, controlling for age, race/ethnicity, and sex, Several of the
tested risk factors were significantly correlated with stroke outcome.
Being a daily smoker, Monthly days with anxiety, and diagnosed
depression were all positively associated with greater odds of stroke.

Daily cigarette smokers had 1.68 times the odds of having had a stroke
than those who have never smoked cigarettes. For each additional day
heavily affected by anxiety,the odds of having had a stroke were 1.03
times higher, and for those who have been diagnosed with depression the
odds of stroke were 2.12 times higher. Of the control variables, age of
25 years old and above were the only factors significantly associated
with stroke outcome, with an odds ratio that increased with age.

\begin{longtable}[]{@{}lll@{}}
\caption{Logistic Regression Model Estimating Effects of Risk Factors \&
Demographic Variables on Stroke Outcome Odds.}\tabularnewline
\toprule
Variable & OR (95\% CI) & P-Value\tabularnewline
\midrule
\endfirsthead
\toprule
Variable & OR (95\% CI) & P-Value\tabularnewline
\midrule
\endhead
\textbf{Risk Factors} & &\tabularnewline
~~~\emph{Drinking} & &\tabularnewline
~~~~~Non-Heavy Drinker & 1 (Baseline Factor) &\tabularnewline
~~~~~Heavy Drinker & 0.989 (0.41 - 2.38) & 0.98\tabularnewline
~~~\emph{Smoking} & &\tabularnewline
~~~~~Non-Smoker & 1 (Baseline Factor) &\tabularnewline
~~~~~Former Smoker & 1.24 (0.91 - 1.70) & 0.18\tabularnewline
~~~~~Smoker (Some days) & 1.65 (0.95 - 2.88) & 0.08\tabularnewline
~~~~~Smoker (Daily) & 1.68 (1.12 - 2.53) & 0.01\tabularnewline
~~~\emph{Mental Health} & &\tabularnewline
~~~~~Monthly Days Anxious & 1.03 (1.01 - 1.04) &
\textless{}0.01\tabularnewline
~~~~~No Depression Diagnosis & 1 (Baseline Factor) &\tabularnewline
~~~~~Depression Diagnosis & 2.12 (1.48 - 3.04) &
\textless{}0.01\tabularnewline
~~~~~Daily Sleep Hours & 0.91 (0.82 - 1.00) & 0.06\tabularnewline
~~~\emph{Sex} & &\tabularnewline
~~~~~Male & 1 (Baseline Factor) &\tabularnewline
~~~~~Female & 0.77 (0.58 - 1.02) & 0.07\tabularnewline
\textbf{Race/Ethnicity} & &\tabularnewline
~~~Other Race / Non-Hispanic & 1 (Baseline Factor) &\tabularnewline
~~~Hispanic & 1.42 (0.43 - 4.70) & 0.57\tabularnewline
~~~Black Only / Non-Hispanic & 2.41 (0.88 - 6.62) & 0.09\tabularnewline
~~~White Only / Non-Hispanic & 1.87 (0.73 - 4.83) & 0.20\tabularnewline
~~~Multiracial / Non-Hispanic & 2.92 (0.90 - 9.50) & 0.08\tabularnewline
\textbf{Age Group} & &\tabularnewline
~~~Age 18 to 24 & 1 (Baseline Factor) &\tabularnewline
~~~Age 25 to 34 & 3.92 (0.80 - 19.66) & 0.10\tabularnewline
~~~Age 35 to 44 & 9.33 (2.00 - 43.51) & \textless{}0.01\tabularnewline
~~~Age 45 to 54 & 12.35 (2.79 - 54.71) & \textless{}0.01\tabularnewline
~~~Age 55 to 64 & 20.93 (4.80 - 91.34) & \textless{}0.01\tabularnewline
~~~Age 65 or Older & 44.87 (10.36 - 194.43) &
\textless{}0.01\tabularnewline
\bottomrule
\end{longtable}

The effect of each significant risk factor, stratified by depression
diagnosis, on the probability of having had a stroke is shown in Figure
2. The probability of a person having a stroke increases with the number
of days a person is significantly affected by anxiety. This probability
is multiplied if a person has been diagnosed with depression. For
non-smokers with no depression diagnosis who did not signifantly suffer
from anxiety, the probability of having had a stroke approaches zero. If
the same subject was diagnosed with depression, the probability of
having had a stroke nears 5\%.

The probability of having had a stroke is multiplied in daily smokers,
beginning at approximately 5\% for daily smokers who have not been
diagnosed with depression nor have suffered from significant anxiety.
With a depression diagnosis, the probability originates at approximately
7\% for subjects with zero days of anxiety and increases to 15\% for
subjects who experienced thirty days of signiticant anxiety. This
relationship is similar for subjects who currently smoke some days.

\begin{figure}
\centering
\includegraphics{report_files/figure-latex/figWithCaption2-1.pdf}
\caption{Effects of Anxious Days on Stroke Outcome, Stratified by
Depression Diagnosis and Smoker Status}
\end{figure}

\section{Conclusion}\label{conclusion}

Our study assessed the association between heavy drinking, anxiety,
depression on stroke odds. We used the 2016 Behavioral Risk Factor
Surveillance System dataset from the CDC. There was a significant
difference between being a daily smoking and odds of a stroke as the
p-value was .01. Also, there was a significant difference for anxiety,
and depression for the odds of a stroke, p-value was less than .01 for
both variables. We found know no significant different between race or
the amount of sleep hours someone receives., and whether or not an
individual is a heavy drinker.

Our study has a specific limitation due to its study design. The study
was a cross-sectional one, causality and risk cannot be inferred from
our data. Each variable tested affected one another, so heavy drinking
influenced the variables anxiety and depression. And anxiety can affect
heavy drinking and depression. The influence of variables on each other
demonstrates that causality cannot be inferred in a cross-sectional
study.

In conclusion, variables depression, anxiety, and being a daily smoker
all were significantly different for stroke odds. This conclusion means
that an individual who is either a daily smoker, diagnosed with
depression or anxiety all have an increased change of having a stroke.
The 2016 BRFSS dataset provided a large population data set to help
monitor variables that affect stroke odds. Future studies can look at
the association can look further at the effect of mental health
diagnosis on stroke odds, as well as how smoking can increase an
individual's odds of a stroke.




\newpage
\singlespacing 
\bibliography{../Data/Misc/bios.bib}

\appendix
\section{Code}
\end{document}